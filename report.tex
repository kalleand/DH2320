\documentclass[a4paper]{article}

\usepackage[T1]{fontenc}
\usepackage[english]{babel}
\usepackage{textcomp}
\usepackage{multirow}
\usepackage{float}
\usepackage{fancyhdr}
\usepackage{pdfpages}
\usepackage{graphicx}
\usepackage{amsmath}

\title{DH2320 Lab work}
\author{Karl Johan Andreasson <{kalleand@kth.se}> %
\and Mikaela N\"oteberg <{mnot@kth.se}>
}

\fancyhf{}
\fancyhead[LE,RO]{\slshape \rightmark}
\fancyhead[LO,RE]{\slshape \leftmark}
\fancyfoot[C]{\thepage}

\begin{document}
\thispagestyle{empty}
\maketitle
\thispagestyle{empty}
\pagestyle{empty}
\newpage
\pagestyle{fancy}
\setcounter{page}{1}

\section{Lab 1: WebGL and Three.js}
\subsection{Make the first cube look like the second one.}

For the first assignment we were supposed to alter how a Javascript using WebGL rendered a
cube to look more like what it did using the Javascript library Three.js.

To do this we changed the position of the cube further away from the perspective
camera located in the origin of the coordinate system. We changed the Z-value
from $-2$ to $-8$, this made the whole cube visible.

This was not enough as we needed a rotation around the y-axis as well as
rotation around the x-axis. To do this we added the snippet:

\begin{verbatim}
mat4.rotate(mvMatrix, rCube/2, [ 0, 1, 0 ]);
\end{verbatim}

\subsection{Adding a tetrahedron}

In the next assignment we added a tetrahedron to both scenes. To accomplish this
we moved the cubes in both scenes to the left by $1.5$ units. Then we added the
Three.js tetrahedron by adding a TetrahedronGeometry with the size of 1.5 units.

The tetrahedron implemented using the vertices:

\begin{verbatim}
(1, 1, -1)
(1, -1, 1)
(-1, 1, 1)
(-1, -1, -1)
\end{verbatim}

The radius of the circumsphere for this tetrahedron is $\approx 1.73$ which is
the value we used when initiating the tetrahedron in the Three.js version.

In the WebGL version this was a lot trickier as we had to figure out where the
vertices where located in a tetrahedron with a certain size.

The rotation around x-axis was set to be the same for the tetrahedron as the
cube. The rotation around z-axis was set to be negative speed of the rotation
around x-axis divided by 2. The tetrahedron, in contrast to the cube, was set to
not rotate around the y-axis.

The colors were fixed to match in both scenes through trial and error mostly.

\section{Lab 2: Photoshop Techniques}

\subsection{Image restoration} % baby, scratched, pink flowers

\subsection{Retouching} % phat

\subsection{Creating a seamless tile} % wood

\subsection{Montage and design} % poster

\subsection{Webpage} % adding structure to cube in three.js

\section{Lab 3: 3D Modelling and animation}

\subsection{A ball bouncing on a table} % Maya

\subsection{Adding lights} % to webpage

\section{Lab 4: Introduction to Data Mining}

%something something, questionnaire?

\section{Lab 5: Introduction to 3D visualization}

\subsection{Visualizing a three-dimensional vector field}

\subsection{Analyzing the air quality of a kitchen}

\subsection{Volume rendering}

\end{document}
